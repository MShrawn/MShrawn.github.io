\section{Summing up}
In this work we looked into the viability of creating a facial recognition system with the purpose 
of tracking classroom attendance.  Doing so, we discovered that there are many ways to improve 
recognition rates of such a system both from general normalization methods, as well as ways to use 
the nature of the problem at hand to help in accuracy. \\

The second aspect we noticed was that there was no readily available system that one could test new 
ideas against existing solutions.  At this point, the purpose of this work changed to creating a 
workstation from which a user could add a new normalization technique (or theorised improvement of 
an already existing method) and compare results to past solutions to the targeted problem.  This work
would then use classroom attendance tracking as an accompanying example.

\subsection{Working Example}
As an illustration of the proposed usage of the system this work looked into the viability of the 
Mean Illumination Estimation in conjunction with the Eigenface method of facial recognition both 
in general and specifically for our needs in classroom tracking.  The experiment probed from two 
directions, one; we compared it against OpenCV's inbuilt histogram equalization solution to 
illumination issues.  Two; we ran the experiment twice, one against the Yale B database and again 
against the AT\&T database. \\

The results of this experiment have already been discussed as can be seen in chapter 3.  However, 
a short run-down of the highlights will follow.  MIE works really well when up against strong 
illumination conditions;  directional lighting, strong global change in ambient light etc.  However, 
it greatly degrades the quality of the image worked on meaning it wouldn't work well with  a multiple 
normalisation technique based solution.  \\

It was noted that OpenCV's histogram equalization would suit our classroom environment more adequately 
than MIE as we deal more with global ambient change as opposed to local lighting changes and histogram 
equalization does not have as much of an effect as MIE on the images processed. \\

Nothing done in this work indicates that a system that uses facial recognition to track attendance in a 
classroom environment is an infeasible goal.  We see from the AT\&T database (one that comprises of faces 
taken under similar conditions to what one would expect in a classroom environment) that even before 
normalization techniques are employed, recognition rates are almost within acceptable margins.  Certainly 
no worse than the inaccuracies of the pen and paper approach commonly employed to solve this problem. \\

A more pressing concern would be the total cost of such a system, on that front, this work envisions that such a
system could work with off-the-shelf webcams.  Computationally, most any hardware system would be sufficient as 
computation time would be of little concern.  Indeed (5 - 10) hours could be spent on computation per day and the 
system would still be viable.
 
\subsection{Theory of future work}
Another aspect to this work was to look into the benefits of using a global optimisation technique 
that would replace Eigenface's greedy first match classification system.  This was looked into in depth 
in chapter 4. \\

Due to limitations with the C++ implementation of OpenCV's Eigenface algorithm, such manipulation of its 
inner workings is currently not available to us.  This is not the only project that has wanted a feature 
like this as indicated in the answer to a request for such functionality to be added by the maintainers of OpenCV
~\cite{Eigenface_classification} who replied that they have had many such requests and that it is possible they 
will be adding such functionality at a later stage. \\

Going with this, this work explored how such data could be exploited to yield better accuracies from systems 
where both the training set and test set are known (or should be known).  Having looked specifically at the 
Hungarian algorithm as it offers a solution in polynomial time of O($n^4$)~\cite{munkres1957algorithms}.  Whereas 
any naive solution to the problem (pick all combinations and choose the smallest) would be of O(n!)

\section{Looking ahead}
This work is such that it will never truly be finished, there are many more normalization techniques that 
can be added into the system for comparison, or one could add different recognition algorithms to classify 
test faces.  Examples would include Fisherfaces and Local Binary Patterns Histograms. \\

Strong candidates for future work are solutions to the assignment problem brought up previously  with the 
provided possible solution being the Hungarian algorithm.  Also needed are some form of alignment normalization 
techniques probably something simple to start like using the eyes to align and scale the face.  Both of these 
are pressing issues that will need to be taken into account if one wishes to create an extensive recognition 
system.  \\

Also now that it is clear that such a system can indeed be useful it could be prudent to change over to a C/C++ 
implementation of the working code, this will improve performance in many areas and provide a more in-depth 
control of OpenCV.  \\

If this system is indeed intended for use in classroom attendance tracking then another system will need to 
be devised to compile a set of images per subject and break them up into class sets at the start of each year, 
also allowing for manual addition or removal of students from such a list.

