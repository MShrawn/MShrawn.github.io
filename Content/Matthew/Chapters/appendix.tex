These are snippets of code that would have been too large to insert within the main body of the thesis.  What is 
seen below is the implementation of the MIE algorithm (A.1) and (A.2).  When coding the method up I paid little 
attention to efficiency.  \\

Looking at the code snippets we see function d() is our main time sink, (A.1) Lines (12 - 24), and probably the 
area we would want to target when attempting to increase efficiency.  As is, the method is a multiplicative kernel 
of size $11 \times 11$ that needs to be passed over each and every pixel.  \\

As we are dealing with a moving kernel, one optimisation that could be made is instead of complete recalculation 
of the kernel for each pixel, if the kernel moves from left to right, subtract (in our case divide) the elements 
from the left of the kernel and multiply the new right hand elements in (There is a similar case when the kernel 
moves down) doing so we change from $11 \times 11 = 121$ calculations per kernel to $11+11=22$ calculations per 
kernel. \\

The code seen in (A.2) contains the main method for the MIE.  As the algorithm is very mathematically inclined there 
is little here but other function calls.  However, we note specifically the suppression method, (A.1) Lines (53 - 77), which 
encapsulates the core workings of the algorithm passes over each pixel in the image and asks if that pixel should be 
suppressed or highlighted.

\newpage
\begin{lstlisting}[language=Python, caption=MIE Implementation Listing 1\label{Listing:1}]
	def d(x, y, img, n):
		# Mathematically
		# d(x, y) = ln((r(x, y))/(prod{(s,t) E w(nxn)}[ r(s,t) ])^(1/n^2))
		
		prod = long(1);
		kernel_size = n/2;
		kernel_cnt = 0;

		# loop through each element in the kernel multiplying them together.
		# component: prod{(s,t) E w(nxn)}[ r(s,t) ]
		
	# =================================================================== #
		for i in range(-kernel_size + 1, kernel_size, 1):		# y
			for j in range(-kernel_size + 1, kernel_size, 1):	# x
			  # make sure we arnt at centre or going out of bounds of img
				if (((y + i) >= 0) 
						and ((y + i) < len(img)) 
						and ((x + j) >= 0) 
						and ((x+j) < len(img[0]))):
					if (img[y + i][x + j] != 0):
						prod *= long(img[y + i][x + j]);
						kernel_cnt+=1;
	
	# =================================================================== #
	
		denominator = (prod**(1/float(kernel_cnt)));
		if img[y][x] == 0:
			pass;
		temp_in = (img[y][x]) / denominator;
		return math.log(temp_in);
		
	def h(x, y, img, alp, beta, n):
		# Mathematically
		# h(x, y)      = e^[d(x,y)/(alpha*beta)]
		return math.exp( d(x, y, img, n)/ (float(alp) * beta));
		
	def Average_grey(img, n):
		# Mathematically
		# Average_grey = 1/(a*b) Sum{(x,y) E f(axb)}[abs(d(x,y))]
		temp_sum = 0;

		# go through each pixel in image and sum the results of 
		# the d function.
		a, b = len(img), len(img[0]);
		for x in range(0, b, 1):        # x
		   for y in range(0, a, 1):     # y
				temp_d = d(x, y, img, n);
				if (temp_d < 0): temp_d = -temp_d;
				temp_sum += temp_d;

		return (1/(float(len(img)*len(img[0]))) * temp_sum);
		
	# =================================================================== #
	def Supression(img, n, beta):					
		# o_hat(x, y)  = 							    	
		#				{ h(x, y);	h(x, y) <  1	
		#				{ 1;		h(x, y) >= 1	   
		
		sup_img     = np.zeros((len(img), len(img[0])));
		# determine average pixel value							
		gry_ave     = Average_grey(img, n);
		c = 1;
		# go through each pixel and determine if it should be suppressed 
		# or highlighted.
		a, b = len(img), len(img[0]);
		for x in range(0, b, 1):        # x
		   for y in range(0, a, 1):     # y
				Highlight = h(x, y, img, gry_ave, beta, n) # cacl value
				if( Highlight < 1):
					sup_img[y][x]         = Highlight;
					if (Highlight < c): c = Highlight;
				else:
					sup_img[y][x]   = 1;

		return sup_img, c;
		
	# =================================================================== #
\end{lstlisting}

\newpage

\begin{lstlisting}[language=Python, caption = MIE Implementation listing 2\label{Listing:2} ]
	def Mean_Illumination_Estimation_single(img):
		# Implementation of the mean illumination estimation algorithm.
		# takes as input a grey-scaled image and returns the Mean
		# illumination estimation version

		img = convertTo(img, 1);
		# tweak-able values
		# Literature determined that the following value gave best 
		# results for beta.
		beta = 2.2
		
		tmp_f   = (len(img[0])/float(168));
		n = int(tmp_f * 11);  	
							# for testing purposes best set on a per-image  
							# bases via the expression:
							# n 	 = [(width/width_o) * n_o]
							# n_0    = 11
							# width  = curr image face width
							# width_0= database standard width for yale=(168)

		# we first perform the suppression.
		# then build up the solution image by creating a blank slate
		# then determine width and height
		# then build up the new image pixel by pixel
		suppressed, c   = Supression(img, n, beta);
		eqImg           = np.zeros((len(img), len(img[0])), dtype=np.uint8);
		a, b            = len(img), len(img[0]);  
		
		for x in range(0, b, 1):          # x
			for y in range(0, a, 1):	  # y
				# Calc to spread out the highlight values over the spectrum
				tmp = int(((255*(suppressed[y][x] - c))/(1-c)));
				eqImg[y][x] = tmp;

		return eqImg;
\end{lstlisting}

