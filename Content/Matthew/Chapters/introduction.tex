\section{Project Background}
	\subsection{Background context}
		Despite modern day technology many school environments still struggle with the problem of class/lecture 
		attendance tracking.  Some may ask, why do we need to track attendance?  Tracking attendance 
		has many useful benefits for schools and universities the obvious one is that many students try skip 
		lectures to avoid work.  Thus tracking their attendance would help in identifying such students.  This 
		would, hopefully, result in greater attendance of such classes/lectures. \\

		The existing solution to this problem has varied slightly but for the most part has either been a simple 
		piece of paper passed around the class letting the students sign/tick their names off (mostly used in 
		universities), or a roll call at the start of the class by teachers in lower level class room environments 
		(primary/high-school).  \\

		Thus it shouldn't come as a surprise that there have been many attempts to solve the problem of lecture 
		attendance tracking and hence remove some issues.  Some of the main ideas put forward are: barcode scanners
		~\cite{barcode}, RFID systems~\cite{RFID} and the ever popular fingerprint scanners~\cite{fingerprintsystems}.  
		The problem with these systems is that they are still all rather intrusive workarounds, requiring students to 
		take an active part in their attendance tracking, this results in either lines outside of lecture room venues 
		as students wait to verify that they are in attendance, or alternatively, a rather distracting procedure to do 
		while they could be listening to the lecture. \\

		Past papers on this topic have addressed the existence of these issues in some context or another
		~\cite{PatelDoaSAMSURaFR,BalcohAfEAM,KarheSASAUFR}.  Now as many have recognized that facial recognition has 
		the potential to be a very simple, and non-intrusive means of tracking attendance, as in the ideal case it
		would simply need a camera at the front of the class and as the lecture goes on it identifies all students 
		present.  However, the technology available today is still not robust enough, hence the need for further 
		research, development and refinement in this field.  \\
		
		Some points of interest to consider are lighting, as it is a very big problem that has had many attempts at a 
		solution most are not satisfactory as they degrade the image too extensively.  The required camera quality is 
		also an area of interest.  Ideally it would be good to get a solution that can work with an off-the-shelf 
		web-cam.  \\		
		
		It should be noted that facial recognition isn't a perfect science to start with.  Many solutions don't even 
		take into account that they are attempting to recognize a face.  These algorithms could be more accurately
		described as object recognizers, some rather popular examples of this type of system include Egienfaces, 
		Fisherfaces. This work takes into account many of these issues and also attempts to use environment specific 
		knowledge to recognize students (the fact that we know both the training set and class set). \\
		
		As stated there have already been numerous attempts at a solution for classroom attendance.  Some have been 
		more successful than others.  However, this work differs from the others as we hope to design a system that 
		can be easily extended or modified to test out new ideas to improve recognition.  Not only do we wish to offer 
		a system one can easily plug new normalization techniques into, but also able to make use of additional 
		information that pertains to a specific problem a user may be facing. \\
		
		Possible additional information that could be exploited include the knowledge that students tend to sit in the 
		same area each day often varying their position by little more than a seat or two.  This knowledge could be used 
		to strengthen accuracy ratings should an individual known to sit at that location is identified.  Another aspect 
		could be useful to such a system is the knowledge that prior to lectures it is already known who should be there.  
		Thus an attempt can be made to optimize the solution between two sets, namely a set of present faces, and the 
		class-list set. \\

		Facial recognition is a complex field and has been well researched over the past decades even so, it is far from 
		being a fully understood or solved problem.  An aspect clearly portrayed by the fact that there are many 
		variations in methods and techniques out there to solve this problem. \\
		
		The work provides an illustrated application of this platform by implementing facial recognition for lecture 
		attendance tracking.  This work focuses on extending the pre-processing side of the tool-kit using the already 
		provided OpenCV implementation of Eigenfaces to do the actual recognition.  One notable addition would be adding 
		the Mean Illumination Estimation algorithm.   Some more concepts that are added include; image cropping, 
		orientation correction and plane alteration.  All of the above concepts describe various aspects of image 
		normalization.
		
	\subsection{Making a Toolbox}
		A constantly developing key aspect of this research project is creating a facial recognition tool-kit.  
		This platform will act as scaffolding for the addition of any features related to facial recognition, be it 
		pre-processing or actual facial recognition algorithms.  The idea is to selectively add relevant image 
		manipulation techniques or other such features to the code base, thus allowing the client to mix and match 
		them and after application get a report stating how successful the combinations that where chosen are.  Some features 
		would be cropping the faces out from the background noise, others would aim to control lighting.   \\
			
			
\section{The Project}
	\subsection{Description of this Work}
		The overall goal of this research is to determine if it is possible to efficiently automate attendance tracking 
		in lectures via facial recognition.  In a related project (not directly part of this work) a video stream of the 
		class is analysed to extract and cluster faces of the same individual.  In this work we take a direct approach that 
		finds the best match for each normalized cluster in the class list of faces.  A second approach is theorised that 
		would simultaneously take into account all possible matchings and find a globally optimized assignment of each 
		cluster to the class list. A comparison of matching accuracies for the two techniques could then be provided to 
		determine the benefits or trade-offs.

	\subsection{What my code should do} 
		My code should be able to: take input of the form of a folder of images.  In each image would be a single face 
		with around ten such images~\cite{OpenCVTut} per subject.  I would then take those images, put them through some form of 
		normalization, thereafter put them through the recognizer which will then compare the created feature vector 
		against those created from the database which was used to train the system.  Doing this comparison, the system 
		would determine the most likely candidate for the subject in question.  

	\subsection{Language choice}
		The language my code will be implemented in will be Python, on a windows based OS, making use of the OpenCV image 
		processing library.  The reason I've chosen Python is for its ease of use with regards to image processing and 
		the intention of this work is that of a working proof of concept so efficiency is not a concern at this stage.  
		I made the choice of using OpenCV as it is the most robust open source library package for image processing.  
		Another factor in favour of OpenCV is that it is open source and thus free.  Part of the aim of this work is to 
		create a low cost system. 
		
	\subsection{Things to note}
		As with most academic work, some formality of raw data is often required, for this work this data is facial images or more 
		accurately, databases of facial images.  I would need such databases to train my system to recognise faces in a 
		controlled manner.  Once I have refined my system to recognise these faces with a satisfactory degree of accuracy 
		that conform with other literature, I will extend it to recognize faces in every day images.  Ultimately this work will be testing using 
		data obtained from a lecture venue environment. \\

		Luckily there is a ready supply of such collections.  A starting point to find such databases would be the site
		~\cite{urldatabases} which catalogues, with descriptions, many such databases.  An initial overview of the site 
		indicates that I should look into the Yale B face Database as it contains far more variety than others especially 
		when it comes to illumination changes.  However, the FERET database makes use of notably more consistent backgrounds.  \\
		
		Another database I will look at is the AT\&T database~\cite{ATTDATA}.  It is smaller than the Yale B database, but has more realistic real 
		world orientations of faces and the more likely accurate representation of images captured in a classroom environment.
	
\section{Other Key Aspects}
	\subsection{Computer Vision}
		For my project I need to know how to recognise faces, hence a useful starting point would be how to interpret images.  
		Both these requirements fall under computer vision.  Computer vision is the field focusing on using image processing 
		to gain information from images for a computer to use.  Ideally this would let the computer view the world in much 
		the same manner as how we humans see it. \\
		
		Computer vision makes use of edge detection, object recognition, feature detection and many other techniques to help 
		make a computer see the world as we do.  This concept is an infinitely harder task than someone outside the field of 
		computer vision may believe.  Mostly due to the effortlessness at which a human brain seems to interpret what it is seeing.
		~\cite{szeliski2010computer}.  More specifically this work will need to understand the interpretation of an image in 
		the Numpy/OpenCV environment.
		
	\subsection{Illumination Normalization}
		Apart from my main aim to verify Faces in images there are certain other considerations to note.  Before I can start 
		to verify faces in images I would need to normalize those images.  Normalizing with respect to image processing is the 
		act of taking images of varying lighting, orientation and alignment and converting them into a semi-consistent form
		~\cite{FDGFN2009}.  \\
		
		Not only will I need to normalise a face but also combine clusters of faces, either finding the best face to use or 
		normalize each face cluster into a single image in an attempt to create an \emph{Average face} for an individual.  
		Another method of using multiple images is to individually normalize and score them all and see whom the majority 
		are classified as.
		
	\subsection{Global/Greedy Optimisation}
		This work made use of OpenCV's eigenface algorithm which makes use of a greedy best pick algorithm.  Such a greedy 
		technique takes a probe face, scores it against all training data and reports on the best match it got, less accurate 
		than other methods but fast. \\
		
		In this work it would be an interesting idea to compare greedy and global based optimising techniques.  A global technique 
		would try find the best fitting of faces such that we have the lowest possible scores linking each face~\cite{GlOP1989}.  
		This would imply we would achieve higher accuracy but possibly a more computationally expensive system.  \\
		
		Clearly a classroom environment where we have a set class-list and know that we should have those students present lends 
		itself nicely to a global optimisation scoring system.  Comparing them would highlight if performing such an optimisation
		provides clear benefits over the in-built greedy match in terms of accuracy at the cost of complexity to implement and 
		computation time. \\
		
		This work explores a global match optimisation idea in chapter four,  but we were not able to implement such a system due to constraints 
		that will be discussed later.